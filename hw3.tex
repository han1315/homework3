\documentclass{ctexart}

\usepackage{graphicx}
\usepackage{amsmath}

\title{作业三: 我的Linux环境说明}


\author{洪晨瀚 \\ 信息与计算科学 3200300133}

\begin{document}

\maketitle
\bibliographystyle{plain}
 
\section{Linux发行版名称及版本号}
\begin{itemize}
\item 名称:Ubuntu
\item 版本号:14.04.6 LTS Ubuntu
\end{itemize}

\section{系统调整及配置}
\begin{itemize}
\item 配置了\verb|emacs|的环境,美化以及能直接输入汉语,配置了字体,字体大小改成15。
\item 下载了\verb|font-manager|管理字体,并下载了许多字体以及从本机中复制字体到虚拟机中。
\item 安装了增强功能,使得功能更多样化。
\end{itemize}

\section{工作规划}
\begin{itemize}
\item 趁着接下来的假期学习\verb|Python|。
\item 之后会尝试用\verb|Ubuntu|编写汇编语言,之前都用\verb|VMware|。
\end{itemize}

\section{使用Linux的场合}
\begin{itemize}
\item 完成作业以及课程。
\item 学习汇编语言。
\end{itemize}

\section{工作环境}
\begin{itemize}
\item 这个系统我已经使用了一个学期,还算满意,至于未来需要更好的环境时会再进行配置,好比这次小学期课程配置许多字体到虚拟机。
\end{itemize}

\section{代码、文献的保存}
\begin{itemize}
\item 不随意调整系统以及系统权限,不更新系统,始终维持当前版本。\cite{de2010new}
\item 将代码以及文献上传到\verb|github|,并复制到移动硬盘中,每次工作后都重复过程,防止数据丢失。
\end{itemize}



\bibliography{han}
\end{document}
